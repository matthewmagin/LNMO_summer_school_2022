%! Author = matheusz
%! Date = 14.07.2021

% Preamble
\documentclass[12pt]{article}

% Packages
\usepackage[left=1.1cm,right=1.1cm,
    top=2cm,bottom=2cm,bindingoffset=0cm]{geometry}
\usepackage{tipa}
\usepackage{euscript}
\usepackage{mathrsfs}
%\usepackage{txfonts}
\usepackage{textcomp}
\usepackage{cmap}
\usepackage{mathtext}
\usepackage[T2A]{fontenc}
\usepackage[english,russian]{babel}
\usepackage{amsfonts}
\usepackage{amsmath,amsfonts,amssymb,amsthm,mathtools} % AMS
\usepackage{icomma} % "Умная" запятая: $0,2$ --- число, $0, 2$ --- перечисление
\usepackage{fancybox,fancyhdr} %this packages provides fancy up and bottom of page
\pagestyle{fancy}
\usepackage{setspace}
%\полуторный интервал
\fancyhead{}
\fancyhead[LE,RO]{\thepage}
\fancyhead[CO]{План курса}
\fancyhead[LO]{Летний лагерь ЛНМО}
\fancyhead[CE]{}
\usepackage{enumitem}
\fancyfoot{}
\usepackage{import}
\usepackage{xifthen}
\usepackage{pdfpages}
\usepackage{transparent}
\usepackage{caption}
\usepackage[condensed,math]{anttor}
\usepackage[T1]{fontenc}


\begin{document}
\begin{center}
    \section*{Теория делимости}
    \subsection*{\textsc{М. Магин}}
\end{center}

\begin{enumerate}
    \item Делимость целых чисел. Определение, базовые свойства.
    \item Простые числа. Теорема Евклида. Теорема о $k$ последовательных составных в натуральном ряде.
    \item Деление с остатком. Существование и единственность остатка. Свойства деления с остатком.
    \item Сравнения по модулю. Определение, основные свойства: арифметика остатков, сокращение на взаимнопростой множитель, сравнение по модулю~--- отношение эквивалентности.
    \item Десятичная запись числа и признаки делимости. Признак делимости на 3 (9), признак делимости на $2^n (5^n)$, признак делимости на 11.
    \item Аксиомы кольца. Примеры и антипримеры колец. Кольцо классов вычетов $\mathbb{Z}/p\mathbb{Z}$.
    \item Наибольший общий делитель, его основные свойства.
    \item Наименьшее общее кратное и его основные свойства.
    \item Алгоритм Евклида. Обобщенный алгоритм Евклида.
    \item Линейное представление НОД. Линейные диофантовы уравнения: критерий разрешимости, общий вид решений.  Лемма Евклида.
    \item Методы решений диофантовых уравнений (на примерах).
    \item Основная теорема арифметики. НОД и НОК в терминах основной теоремы арифметики.
    \item Функция количества делителей $\tau$, функция суммы делителей $\sigma$, степень вхождения простого в факториал.
\end{enumerate}

\end{document}
