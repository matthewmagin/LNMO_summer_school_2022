%! Author = user
%! Date = 11.07.2022

% Preamble
\documentclass[11pt]{article}

% Packages
%----------------------------------------------------------------------------------------
%	PACKAGES AND OTHER DOCUMENT CONFIGURATIONS
%----------------------------------------------------------------------------------------

\usepackage{listings}
\usepackage{graphicx}
\usepackage{booktabs}
\usepackage{enumitem}
\usepackage[left=2cm,right=2cm,
    top=2cm,bottom=2cm,bindingoffset=0cm]{geometry}
\usepackage[utf8]{inputenc}
\usepackage[english,russian]{babel}
\usepackage{titling}
\usepackage{textcomp}
\usepackage{mathtext}
\usepackage{amsmath,amsfonts,amssymb,amsthm,mathtools}
\usepackage{icomma}
\usepackage{import}
\usepackage{amssymb, amsmath}
\usepackage{indentfirst}
\usepackage{moresize}
\usepackage{multicol}
\usepackage{dsfont}
\usepackage{xifthen}
\usepackage{pdfpages}
\usepackage{transparent}
\usepackage{caption}
\usepackage{epigraph}
\usepackage{xcolor}

\newtheorem{statement}{Statement}
\newtheorem{corollary}{Corollary}
\newtheorem{theorem}{Theorem}
\newtheorem{definition}{Definition}
\newtheorem{lemma}{Lemma}
\newtheorem{example}{Example}
\theoremstyle{remark}
\newtheorem{remark}{Remark}
\newtheorem{prop}{Property}




\numberwithin{equation}{section} % Number equations within sections (i.e. 1.1, 1.2, 2.1, 2.2 instead of 1, 2, 3, 4)
\numberwithin{figure}{section} % Number figures within sections (i.e. 1.1, 1.2, 2.1, 2.2 instead of 1, 2, 3, 4)
\numberwithin{table}{section} % Number tables within sections (i.e. 1.1, 1.2, 2.1, 2.2 instead of 1, 2, 3, 4)

\setlength\parindent{0pt} % Removes all indentation from paragraphs

\setlist{noitemsep} % No spacing between list items

%%% Операторы всякие:
\DeclareMathOperator{\arsec}{arsec}
\DeclareMathOperator{\arcsch}{arcsch}
\DeclareMathOperator{\arcosh}{arcosh}
\DeclareMathOperator{\arsinh}{arsinh}
\DeclareMathOperator{\artanh}{artanh}
\DeclareMathOperator{\arsech}{arsech}
\DeclareMathOperator{\grad}{grad}
\DeclareMathOperator{\Log}{Log}
\DeclareMathOperator{\Arg}{Arg}
\renewcommand{\Im}{\mathop{\mathrm{Im}}\nolimits}
\renewcommand{\Re}{\mathop{\mathrm{Re}}\nolimits}
\DeclareMathOperator{\arcoth}{arcoth}
\usepackage{setspace}
%%% Колонтитулы
\usepackage{fancyhdr}
\pagestyle{fancy}
\renewcommand{\sectionmark}[1]{\markright{\thesection\ #1}}

\fancyhead[LE,RO]{\thepage}
\fancyhead[LO]{\rightmark}
\fancyhead[RE]{\leftmark}


%----------------------------------------------------------------------------------------
%	SECTION TITLES
%----------------------------------------------------------------------------------------

%\sectionfont{\vspace{6pt}\centering\normalfont\scshape} % \section{} styling
%\subsectionfont{\normalfont\bfseries} % \subsection{} styling
%\subsubsectionfont{\normalfont\itshape} % \subsubsection{} styling
%\paragraphfont{\normalfont\scshape} % \paragraph{} styling

\newcommand{\RNumb}[1]{\uppercase\expandafter{\romannumeral #1\relax}}

\renewcommand\thesection{\arabic{section}.}
\renewcommand\thesubsection{\thesection\arabic{subsection}}
\renewcommand\thesubsubsection{\thesubsection.\arabic{subsubsection}}
\renewcommand{\bf}{\textbf}
%----------------------------------------------------------------------------------------
%	HEADERS AND FOOTERS
%----------------------------------------------------------------------------------------
\DeclareMathOperator{\ord}{ord}
\DeclareMathOperator{\ld}{ld}
\DeclareMathOperator{\exi}{exi}
\DeclareMathOperator{\num}{num}
\DeclareMathOperator{\den}{den}
\DeclareMathOperator{\diam}{diam}
\DeclareMathOperator{\sign}{sign}
\DeclareMathOperator{\len}{len}
\DeclareMathOperator{\vp}{v.p.}
\DeclareMathOperator{\osc}{osc}
\newcommand{\divisible}{\mathop{\raisebox{-2pt}{\vdots}}}
\DeclareRobustCommand{\divby}{%
     \mathrel{\text{\vbox{\baselineskip.65ex\lineskiplimit0pt\hbox{.}\hbox{.}\hbox{.}}}}%
}
\newcommand{\eqdef}{\stackrel{\mathrm{def}}{=}}
\DeclareRobustCommand{\notdivby}{%
     \!\!\not\;\divby%
}
\DeclareMathOperator{\Int}{Int}
\DeclareMathOperator{\Cl}{Cl}
\DeclareMathOperator{\Fr}{Fr}
\newcommand{\Mod}[1]{\ (\mathrm{mod}\ #1)}



\addto{\captionsrussian}{\renewcommand{\abstractname}{АННОТАЦИЯ}}

\newcommand{\incfig}[1]{%
    \def\svgscale{1.5}
    \import{./figures/}{#1.pdf_tex}
}
\graphicspath{{pictures/}}
\DeclareGraphicsExtensions{.pdf,.png,.jpg, .jpeg, .tex}

\definecolor{myblue}{RGB}{72, 184, 178}
\definecolor{myblue1}{RGB}{0, 109, 167}
\usepackage{color}
\usepackage[colorlinks,urlcolor = blue, filecolor=blue,citecolor=blue, linkcolor = blue]{hyperref}

\begin{document}
\begin{center}
    \section*{Теория делимости}
    \subsection*{\textsc{М. И. Магин, Б.А. Золотов}}
\end{center}

\begin{enumerate}
    \item Делимость целых чисел. Определение, базовые свойства делимости. Свойства четных и нечетных чисел.
    \item Простые числа. Теорема Евклида. Теорема о $k$ последовательных составных в натуральном ряде.
    \item Деление с остатком. Существование и единственность остатка.
    \item Сравнения по модулю. Определение, основные свойства: арифметика остатков, сокращение на взаимнопростой множитель, сравнимость по модулю~--- отношение эквивалентности.
    \item Десятичная запись числа и признаки делимости. Признак делимости на 3 (9), признак делимости на 11.
    \item Признак делимости на $2^n (5^n)$.
    \item Аксиомы кольца. Примеры и антипримеры колец. Кольцо классов вычетов $\mathbb{Z}/p\mathbb{Z}$ (определение).
    \item Наименьшее общее кратное. Свойства НОК: любое общее кратное набора чисел делится на НОК, $\lcm(a_1, \ldots, a_n) = \lcm(\lcm(a_1, \ldots, a_{n - 1}), a_n)$.
    \item Наибольший общий делитель. Свойства НОД: НОД набора чисел делится на любой общий делитель, $\gcd(a_1, \ldots, a_n) = \gcd(\gcd(a_1, \ldots, a_{n - 1}), a_n)$.
    \item Свойства НОД: $\lcm(a, b) \cdot \gcd(a, b) = a \cdot b$, $\gcd(a, b) = \gcd(a, a + b) = \gcd(a, a - b)$.
    \item Алгоритм Евклида.
    \item Обобщенный алгоритм Евклида. Линейное представление НОД.
    \item Линейные диофантовы уравнения: критерий разрешимости, общий вид решений.
    \item Методы решений диофантовых уравнений: перебор с отсечениями, метод спуска, разложение на множители.
    \item Лемма Евклида. Основная теорема арифметики.
    \item НОД и НОК в терминах основной теоремы арифметики. Формула для функции $\tau$ количества делителей.
    \item Формула для функции суммы делителей $\sigma$, степень вхождения простого в факториал.
    \item Число сочетаний $\binom{n}{k}$. Бином Ньютона, доказательство по индукции.
    \item Лемма о $(a + b)^{p} \equiv a^p + b^p \pmod{p}$. Малая теорема Ферма, доказательство через лемму.
    \item Китайская теорема об остатках.
    \item Пример применения КТО.
    \item Коэффициенты разложения по исходным числам в алгоритме Евклида: метод \lq\lq сверху вниз\rq\rq.
    \item Определение функции Эйлера. Четность. Значения для простого числа и степени простого числа.
    \item Мультипликативность функции Эйлера. Явная формула для функции Эйлера.
    \item В случае конечного \(G\) сократимость равносильна существованию обратного в определении группы.
    \item Группа \(V(n)\) остатков, взаимно простых с \(n\). Теорема Эйлера.
    \item Длина цикла остатков при возведении в степень, когда основание не взаимно просто с модулем.
    \item \(\sum_{d \mid n} \varphi(d) = n\).
\end{enumerate}
\pagestyle{empty}
\end{document}

