%! Author = Matthew
%! Date = 21.06.2022

% Preamble
\documentclass[11pt]{article}

% Packages
%----------------------------------------------------------------------------------------
%	PACKAGES AND OTHER DOCUMENT CONFIGURATIONS
%----------------------------------------------------------------------------------------

\usepackage{listings}
\usepackage{graphicx}
\usepackage{booktabs}
\usepackage{enumitem}
\usepackage[left=2cm,right=2cm,
    top=2cm,bottom=2cm,bindingoffset=0cm]{geometry}
\usepackage[utf8]{inputenc}
\usepackage[condensed,math]{anttor}
\usepackage[T1]{fontenc}
\usepackage[english,russian]{babel}
\usepackage{titling}
\usepackage{textcomp}
\usepackage{mathtext}
\usepackage{amsmath,amsfonts,amssymb,amsthm,mathtools}
\usepackage{icomma}
\usepackage{import}
\usepackage{amssymb, amsmath}
\usepackage{indentfirst}
\usepackage{moresize}
\usepackage{multicol}
\usepackage{dsfont}
\usepackage{xifthen}
\usepackage{pdfpages}
\usepackage{transparent}
\usepackage{caption}
\usepackage{epigraph}
\usepackage{xcolor}

\newtheorem{statement}{Statement}
\newtheorem{corollary}{Corollary}
\newtheorem{theorem}{Theorem}
\newtheorem{definition}{Definition}
\newtheorem{lemma}{Lemma}
\newtheorem{example}{Example}
\theoremstyle{remark}
\newtheorem{remark}{Remark}
\newtheorem{prop}{Property}




\numberwithin{equation}{section} % Number equations within sections (i.e. 1.1, 1.2, 2.1, 2.2 instead of 1, 2, 3, 4)
\numberwithin{figure}{section} % Number figures within sections (i.e. 1.1, 1.2, 2.1, 2.2 instead of 1, 2, 3, 4)
\numberwithin{table}{section} % Number tables within sections (i.e. 1.1, 1.2, 2.1, 2.2 instead of 1, 2, 3, 4)

\setlength\parindent{0pt} % Removes all indentation from paragraphs

\setlist{noitemsep} % No spacing between list items

%%% Операторы всякие:
\DeclareMathOperator{\arsec}{arsec}
\DeclareMathOperator{\arcsch}{arcsch}
\DeclareMathOperator{\arcosh}{arcosh}
\DeclareMathOperator{\arsinh}{arsinh}
\DeclareMathOperator{\artanh}{artanh}
\DeclareMathOperator{\arsech}{arsech}
\DeclareMathOperator{\grad}{grad}
\DeclareMathOperator{\Log}{Log}
\DeclareMathOperator{\Arg}{Arg}
\renewcommand{\Im}{\mathop{\mathrm{Im}}\nolimits}
\renewcommand{\Re}{\mathop{\mathrm{Re}}\nolimits}
\DeclareMathOperator{\arcoth}{arcoth}
\usepackage{setspace}
%%% Колонтитулы
\usepackage{fancyhdr}
\pagestyle{fancy}
\renewcommand{\sectionmark}[1]{\markright{\thesection\ #1}}

\fancyhead[LE,RO]{\thepage}
\fancyhead[LO]{\rightmark}
\fancyhead[RE]{\leftmark}


%----------------------------------------------------------------------------------------
%	SECTION TITLES
%----------------------------------------------------------------------------------------

%\sectionfont{\vspace{6pt}\centering\normalfont\scshape} % \section{} styling
%\subsectionfont{\normalfont\bfseries} % \subsection{} styling
%\subsubsectionfont{\normalfont\itshape} % \subsubsection{} styling
%\paragraphfont{\normalfont\scshape} % \paragraph{} styling

\newcommand{\RNumb}[1]{\uppercase\expandafter{\romannumeral #1\relax}}

\renewcommand\thesection{\arabic{section}.}
\renewcommand\thesubsection{\thesection\arabic{subsection}}
\renewcommand\thesubsubsection{\thesubsection.\arabic{subsubsection}}
\renewcommand{\bf}{\textbf}
%----------------------------------------------------------------------------------------
%	HEADERS AND FOOTERS
%----------------------------------------------------------------------------------------
\DeclareMathOperator{\ord}{ord}
\DeclareMathOperator{\ld}{ld}
\DeclareMathOperator{\exi}{exi}
\DeclareMathOperator{\num}{num}
\DeclareMathOperator{\den}{den}
\DeclareMathOperator{\diam}{diam}
\DeclareMathOperator{\sign}{sign}
\DeclareMathOperator{\len}{len}
\DeclareMathOperator{\vp}{v.p.}
\DeclareMathOperator{\osc}{osc}
\newcommand{\divisible}{\mathop{\raisebox{-2pt}{\vdots}}}
\DeclareRobustCommand{\divby}{%
     \mathrel{\text{\vbox{\baselineskip.65ex\lineskiplimit0pt\hbox{.}\hbox{.}\hbox{.}}}}%
}
\newcommand{\eqdef}{\stackrel{\mathrm{def}}{=}}
\DeclareRobustCommand{\notdivby}{%
     \!\!\not\;\divby%
}
\DeclareMathOperator{\Int}{Int}
\DeclareMathOperator{\Cl}{Cl}
\DeclareMathOperator{\Fr}{Fr}
\newcommand{\Mod}[1]{\ (\mathrm{mod}\ #1)}



\addto{\captionsrussian}{\renewcommand{\abstractname}{АННОТАЦИЯ}}

\newcommand{\incfig}[1]{%
    \def\svgscale{1.5}
    \import{./figures/}{#1.pdf_tex}
}
\graphicspath{{pictures/}}
\DeclareGraphicsExtensions{.pdf,.png,.jpg, .jpeg, .tex}

\definecolor{myblue}{RGB}{72, 184, 178}
\definecolor{myblue1}{RGB}{0, 109, 167}
\usepackage{color}
\usepackage[colorlinks,urlcolor = blue, filecolor=blue,citecolor=blue, linkcolor = blue]{hyperref}

% Document
\begin{document}
\begin{center}
\section{Теория чисел}
\end{center}
\subsection{Делимость и ее свойства:}

В этом параграфе все числа целые, если иного не оговорено.
\begin{definition}
    Число $a$ \emph{делится} на число $b \neq 0$ ($a \divisible b$), если существует такое число $c$, что $a = b \cdot c$.\\
	В этом случае говорят, что $b$~--- \emph{делитель} числа $a$.
\end{definition}

\textbf{Базовые факты, связанные с делимостью:}
\begin{enumerate}
	\item $a \divisible m$ и $b \divisiblem$ $\Longrightarrow$ $(a \pm b) \divisible m$, $ab \divisible m$.
	\item $a \divisible m$  и $b$ $\divisible m$ $\Longrightarrow$ $\forall k, l \in \mathbb{Z} $ $(ka \pm lb) \divisible m$
	\item $a \divisible m$ и $b \not\divisible m$  $\Leftrightarrow$ $(a \pm b) \not\divisible m \Longrightarrow (a \pm b)$ $\not\divisible$ $m$
	\item $a \divisible m$ и $m \divisible k$ $\Longrightarrow$ $a$ $\divisible$ $k$
\end{enumerate}
\begin{proof} Всё это доказывается как-то так:
	\[a \divisible m \Leftrightarrow a = q \cdot m, q \in \mathbb{Z}, \ b \divisible m \Leftrightarrow b = p \cdot m, p \in \mathbb{Z} \Rightarrow a \pm b = q \cdot m \pm p \cdot m = m \cdot (p + q) \Leftrightarrow (a \pm b) \divisible m\]


\end{proof}

\begin{example}
Найдите все такие натуральные числа $a$, что $\frac{2a + 1}{a - 2}$ - целое число.
\end{example}
\textit{Решение:}
\noindent$\frac{2a + 1}{a - 2} \in \mathbb{Z} \Leftrightarrow (2a + 1) \divisible (a - 2)$, а значит и разность этих чисел делится на $(a - 2)$.\\
То есть, $((2a + 1) - (a - 2)) \divisible (a - 2) \Leftrightarrow (a + 3) \divisible (a - 2)$.\\
 Кроме того, разность этого числа и $(a - 2)$ также должна делиться на  $(a - 2)$, то есть $((a + 3) - (a - 2)) \divisible (a - 2) \Leftrightarrow 5 \divisible (a - 2)$.\\
То есть, $(a - 2)$~--- делитель числа 5, а значит он может быть равен $1, -1, 5, -5$. Переберем все случаи, так как их не так много:

\begin{enumerate}
	\item $a - 2 = -1 \Leftrightarrow a = 1$. $\frac{2 a + 1}{a - 2} = \frac{2 + 1}{1 - 2} = -3 \in \mathbb{Z}$, а значит $a = 1$ подходит.

	\item $a - 2 = 1 \Leftrightarrow a = 3$. $\frac{2 a + 1}{a - 2} = \frac{6 + 1}{3 - 2} = 7 \in \mathbb{Z}$, а значит $a = 3$ подходит.

	\item $a - 2 = -5 \Leftrightarrow a = -3$, но $a \in \mathbb{N}$, а значит этот случай не подходит.

	\item $a - 2 = 5 \Leftrightarrow a = 7$. $\frac{2 a + 1}{a - 2} = \frac{14 + 1}{7 - 2} = 3 \in \mathbb{Z}$, а значит $a = 7$ подходит.
\end{enumerate}

\textit{Ответ:} $\left\{ 1, 3, 7\right\}$.\\


\textbf{Свойства четных и нечетных чисел:}

\begin{enumerate}
	\item Сумма двух последовательных натуральных чисел~--- нечетное число.

	\item Сумма четного и нечетного чисел~--- нечетное число.

	\item Сумма любого количества четных чисел~--- четное число.

	\item Сумма четного количества нечетных чисел~--- четное число, в то время как сумма нечетного количества нечетных чисел - нечетное число.

	\item Произведение двух последовательных натуральных чисел~--- четное число.

\end{enumerate}

\begin{theorem}

	 Произведение двух последовательных чисел делиться на 2\\

\end{theorem}


\begin{example}

	В ряд выписаны числа от 1 до 10. Можно ли расставить между ними знаки <<$+$>> и <<$-$>> так, чтоб значение полученного выражения было равно нулю?

\end{example}

\textit{Решение:}

Среди чисел от 1 до 10 имеется ровно 5 нечетных чисел, а значит, их сумма, вне зависимости от того, с каким знаком их брать, будет нечетным числом, а значит и сумма всех чисел будет нечетным числом. То есть, нулем она быть не может, так как ноль - четное число.\\

\textit{Ответ:} нет.

\begin{definition}
    Число $p \in \mathbb{N}$ называется \emph{простым}, если $p > 1$ и $p$ не имеет положительных делитетлей, отличных от $1$ и $p$.
\end{definition}

\begin{statement}
	Если $p_1$ и $p_2$~--- простые числа и $p_1 \divisible p_2$, то $p_1 = p_2$.
\end{statement}

\begin{theorem}[Евклид]

    Множество простых чисел счетно.

\end{theorem}

\begin{proof}
	\textcolor{magenta}{\texttt{Будет добавлено.}}

\end{proof}

\begin{definition}

	Натуральное число $n \mathbb{Z}$ называется составным, если оно имеет хоть один положительный делитель, отличный от $1$ и $n$.

\end{definition}

\begin{remark}

	Число 1 не является ни простым, ни составным.

\end{remark}

\subsubsection{Деление с остатком}

\begin{definition}
    Пусть a и b $\neq 0$~--- два целых числа. Разделить число  a на число b с остатком~--- значит найти такие целые числа $q$ и $r$, что выполнены следующие условия:

	\begin{enumerate}

	    \item $a = bq + r$

		\item $0 \le r < |b|$

	\end{enumerate}

	При этом число $q$ называется \emph{неполным частным}, а число $r$~--- \emph{остатком от деления числа $a$ на $b$}.

\end{definition}

\begin{theorem}

    Для любых $a, b \in \mathbb{Z}$ существуют единственные $q, r \in \mathbb{Z}, \ 0 \le r < |b|$, что $a = bq + r$.

\end{theorem}

\begin{proof}

    \textcolor{magenta}{\texttt{Будет написано.}}

\end{proof}

\begin{corollary}

	Пусть $m \in \mathbb{N}, \ m > 1$. Каждое целое число при делении на $m$ даёт некоторый остаток, причем остатков ровно $m$.
	Это могут быть числа $0, 1, \ldots, m - 1$.

\end{corollary}

Рассмотрим некоторые элементарные методы вычисления остатка.

\begin{theorem}

	Сумма (произведение) чисел $a$ и $b$ дает тот же остаток при делении на число  $m$, что и сумма (произведение) остатков чисел a и b при делении на число $m$.

\end{theorem}

\begin{example}
	Найдите остаток числа $2^{2012}$ при делении на 3.
\end{example}
\textit{Решение:}\\
Заметим, что $2^{2012} = 4^{1006}$. Число 4  дает остаток 1 при делении на 3, а значит по теореме выше (о произведении остатков), $4^k$ даст остаток $1^k = 1$.

\textit{Ответ:} 1.

\subsubsection{Сравнения по модулю}
\begin{definition}
	Если целые числва $a$ и $b$ при делении на натуральное число $m$ дают равные остатки, то говорят, что \emph{a сравнимо с b по модулю m} и пишут $a \equiv b \pmod m$.
Иначе говоря, такая запись означает, что $(a - b)$ делится на $m$.
\end{definition}

При помощи таких обозначений громоздкое предложение <<а дает остаток b при делении на c>> теперь можно записать, как $a \equiv b \pmod c$.\\

На мой взгляд, работать с остатками в целом гораздо проще при помощи сравнений по модулю. У сравнений есть множество полезных свойств, рассмотрим самые основные:\\

\textbf{Основные свойства сравнений по модулю:}
\begin{enumerate}
	\item Арифметические действия:\\
	$\begin{cases}
	a \equiv b \pmod m \\
	c \equiv d \pmod m
	\end{cases} \Longrightarrow \left . \begin{array}{l}(a \pm c) \equiv (b \pm d) \pmod m \\ a \cdot c \equiv b \cdot d \pmod m \end{array} \right .$
	\item Возведение в степень:\\
	$a \equiv b \pmod m \Longrightarrow a^k \equiv b^k \pmod m$
	\item
	Перенос в другую часть равенства:\\
	$(a + b) \equiv c \pmod m \Longrightarrow a \equiv (c - b) \pmod m$
	\item Транзитивность:\\
	$\begin{cases} a \equiv b \pmod m \\ b \equiv c \pmod m\end{cases} \Longrightarrow a \equiv c \pmod m$
\end{enumerate}
\begin{proof}

	\textcolor{magenta}{\texttt{Будет добавлено.}}

\end{proof}

\begin{statement}

    Сравнимость мо модулю~--- отношение эквивалентности.

\end{statement}
\begin{proof}

    \textcolor{magenta}{\texttt{Будет добавлено.}}

\end{proof}

\subsection{Десятичная запись числа. Признаки делимости.}

\begin{definition}
    Любое натуральное число представимо в виде:
\[ n = \overline{a_{k}a_{k - 1}\ldots a_0} = a_k \cdot 10^k + a_{k - 1} \cdot 10^{k -1} + \ldots + a_0 \]
Такая запись называется десятичной записью числа n.
\end{definition}

\begin{example}
	Двузначное число умножили на произведение его цифр, в результате чего получилось трехзначное число, состоящее из одинаковых цифр, совпадающих с последней цифрой исходного числа. Найдите исходное число.
\end{example}
\textit{Решение:}
Обозначим исходное двузначное число за $\overline{ab}$.\\
Теперь мы можем переписать условие задачи в виде уравнения:
\[ \overline{ab} \cdot (ab) = \overline{bbb} \Leftrightarrow
(10a + b) \cdot ab = 100b + 10b + b \Leftrightarrow
10a^2b + ab^2 = 111b \]
Ясно, что при $b = 0$ условие не выполняется. Если $b \neq 0$, то на него можно поделить обе части:
\[10a^2 + ab = 111 \Leftrightarrow ab = 111 - 10a^2\]
Так как $ab > 0$ , $10a^2 < 111$, а значит $a$ либо 1, либо 2, либо 3. Рассмотрим случаи по порядку.
\begin{itemize}
	\item Если $a = 1$, $b = 101$, а это невозможно, так как $b$~--- цифра.
	\item Если $a = 2$, $b = \frac{71}{2}$, а это невозможно, так как $b$~--- цифра.
	\item Если $a = 3$, $b = 7$. Тогда, искомое число~--- 37.
\end{itemize}
\textit{Ответ:} 37.

\begin{example}
	Найдите все натуральные числа, являющиеся степенью двойки, при зачеркивании первой цифры у которых получается число, также являющееся степенью двойки.
\end{example}
\textit{Решение:}
Пусть мы зачеркнули первую цифру числа $2^n$, состоящего из $k + 1$ цифр. Тогда  $10^k < 2^n < 10^{k + 1}$, $10^{k - 1} < 2^m < 10^k$, а значит $\cfrac{1}{10^k} < \cfrac{1}{2^m} < \cfrac{1}{10^{k - 1}}.$\\
Если перемножить первое и третье неравенства, то получится, что $1 < 2^{n - m} < 10^{2} \Longleftrightarrow$\\ $0 < n - m < 8$.\\
Так как цифру заканчивали слева, $2^n$ и $2^m$ заканчиваются на одну и ту же цифру, а значит:\\
\[2^n - 2^m \equiv 0 \pmod{10} \Leftrightarrow 2^m(2^{n - m} - 1) \equiv 0 \pmod{10} \Leftrightarrow
2^{n - m} - 1 \equiv 0 \pmod 5 \Leftrightarrow 2^{n - m} \equiv 1 \pmod 5 \]

Рассмотрим таблицу остатков от деления $2^n$ на 5:

\begin{center}
\begin{tabular}{ | l | l | l | l | l | l | l | l | l |}
\hline
$2^n$ & 2 & 4 & 8 & 16 & 32 & 64 & 128 & \ldots \\ \hline
Остаток от деления $2^n$ на 5   & 2 & 4 & 3 & 1 & 2 & 4 & 3 & \ldots\\\hline
\end{tabular}
\end{center}

Учитывая при этом $1 < n - m < 8 $, $n - m = 4 \Leftrightarrow m = n - 4$.\\
Обозначим зачеркнутую цийру числа $2^n$ за $a$. Тогда
\[2^n - a \cdot 10^k = 2^{n - 4} \Leftrightarrow
2^{n - 4} \cdot (2^{4} - 1) = a \cdot 10^k \Leftrightarrow
2^{n - 4} \cdot 3 \cdot 5 = a \cdot 2^k \cdot 5^k \]
Так как в левой части всего одна пятерка, $k = 1$, а значит, искомое число двузначное.\\
Перебирая все двузначные степени двойки, понимаем, что подходят числа 32 и 64.\\
\textit{Ответ:} 32, 64.

\textbf{Признаки делимости натуральных чисел:}

\begin{theorem} \textbf{(Признак делимости на 5)}\\
	Число $a$ делится на $5$ тогда и только тогда, когда последние цифры десятичной записи числа $a$~--- это 0 или 1.
\end{theorem}
\begin{proof}
    \textcolor{magenta}{\texttt{Будет добавлено.}}
\end{proof}

\begin{theorem}\textbf{(Признак делимости на 3 и на 9)}\\
	Число $a \in \mathbb{Z}$ даёт такой же остаток от деления на 3 (и на 9), что и сумма цифр числа $a$.
\end{theorem}

\begin{proof}

	Пусть $\overline{a_na_{n - 1}\ldots a_2 a_1}$ -- десятичная запись данного числа $a$, то есть
	\[a = \overline{a_na_{n - 1}\ldots a_2 a_1} = a_n \cdot 10^{n - 1} + a_{n - 1}\cdot 10^{n - 2} + \ldots a_1 \cdot 10^0 \]
	Так как $10 \equiv 1 \pmod 3$, $10^{i} \equiv 1^{i} \pmod 3 \equiv 1 \pmod 3 \Rightarrow a_i \cdot 10^{i - 1} \equiv a_i \pmod 3$.  \\
	Применим это к каждому слагаемому и сложим все, получим:\\
	\[a_n \cdot 10^{n - 1} + a_{n - 1}\cdot 10^{n - 2} + \ldots a_1 \cdot 10^0 \equiv (a_n + a_{n - 1} + \ldots  a_2 + a_1) \pmod 3\]
	Так как $10 \equiv 1 \pmod{9}$, аналогичное доказательство проходит и для $9$.

\end{proof}

\begin{example}
	
	Два числа отличаются перестановкой цифр. Может ли их разность быть равной 20072008?
	
\end{example}
\textit{Решение:}\\
Как мы помним, сумма цифр числа дает тот же остаток от деления на 9, что и само число. Значит, разность описанных в условии задачи чисел должна делиться на 9, так как у этих чисел одинаковая сумма цифр:\\
Пусть первое число - a, $a \equiv c \pmod 9$, второе число $b$, $b \equiv c \pmod 9$.\\
$\begin{cases}a \equiv c \pmod 9 \\ b \equiv c \pmod 9 \end{cases} \Longrightarrow a - b \equiv c -c \pmod 9 \Longleftrightarrow a - b \equiv 0 \pmod 9 \Longleftrightarrow (a - b)$ $\divisible$ $9$. \\
Но, $20072008$ $\not\divisible$ $9$, а значит это невозможно.\\

\begin{theorem} \textbf{(Признаки делимости на 4 и на 8)}
	Число делится на 4 тогда и только тогда, когда две его последние цифры составляют число, которое делится на 4. Число делится на 8 тогда и только тогда, когда три его последние цифры составляют число, которое делится на 8.
\end{theorem}
\begin{proof}
    \textcolor{magenta}{\texttt{Будет дописано.}}
\end{proof}
\begin{theorem} \textbf{(Признак делимости на 11)}
    Число делится на 11 тогда и только тогда, когда разность суммы цифр, стоящих на нечетных местах и суммы цифр, стоящих на четных местах, делится на 11.
\end{theorem}
\begin{proof}
    \textcolor{magenta}{\texttt{Будет дописано.}}
\end{proof}

\begin{example}
	Рассмотрим число 305792608. $(8 + 6 + 9 + 5 + 3) - (0 + 2 + 7 + 0) = 22$ $\divisible$ $11$, а значит $305792608$ $\divisible$ $11$.
\end{example}

\begin{example}

	На доске написано  такое число: $72x3y$, где $x$ и $y$ - некоторые цифры. Замените звездочки цифрами так]6 чтобы полученное число делилось на 45.

\end{example}
\textit{Решение:}
Так как число должно делиться на 45, оно должно делиться на 5 и на 9 соответсвенно.
Так как оно делится на 5, его последняя цифра либо 0, либо 5, а значит либо $y = 0$, либо $y = 5$. \\
Так как число делится на 9, сумма его цифр должна делиться на 9. Рассмотрим сумму цифр числа:
\[ (7 + 2 + x + 3 + y) \divisible 9 \Leftrightarrow (x + y + 12) \divisible 9\]
$y = 5\colon \ (x + 17)$ $\divisible$ $9$, а значит $x = 1$.\\
$y = 0\colom \ (x + 12)$ $\divisible$ $9$, а значит $x = 6$.\\

\end{document}