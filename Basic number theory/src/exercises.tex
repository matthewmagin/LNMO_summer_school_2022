%! Author = user
%! Date = 27.06.2022

% Preamble
\documentclass[11pt]{article}

% Packages
%----------------------------------------------------------------------------------------
%	PACKAGES AND OTHER DOCUMENT CONFIGURATIONS
%----------------------------------------------------------------------------------------

\usepackage{listings}
\usepackage{graphicx}
\usepackage{booktabs}
\usepackage{enumitem}
\usepackage[left=2cm,right=2cm,
    top=2cm,bottom=2cm,bindingoffset=0cm]{geometry}
\usepackage[utf8]{inputenc}
\usepackage[english,russian]{babel}
\usepackage{titling}
\usepackage{textcomp}
\usepackage{mathtext}
\usepackage{amsmath,amsfonts,amssymb,amsthm,mathtools}
\usepackage{icomma}
\usepackage{import}
\usepackage{amssymb, amsmath}
\usepackage{indentfirst}
\usepackage{moresize}
\usepackage{multicol}
\usepackage{dsfont}
\usepackage{xifthen}
\usepackage{pdfpages}
\usepackage{transparent}
\usepackage{caption}
\usepackage{epigraph}
\usepackage{xcolor}

\newtheorem{statement}{Statement}
\newtheorem{corollary}{Corollary}
\newtheorem{theorem}{Theorem}
\newtheorem{definition}{Definition}
\newtheorem{lemma}{Lemma}
\newtheorem{example}{Example}
\theoremstyle{remark}
\newtheorem{remark}{Remark}
\newtheorem{prop}{Property}




\numberwithin{equation}{section} % Number equations within sections (i.e. 1.1, 1.2, 2.1, 2.2 instead of 1, 2, 3, 4)
\numberwithin{figure}{section} % Number figures within sections (i.e. 1.1, 1.2, 2.1, 2.2 instead of 1, 2, 3, 4)
\numberwithin{table}{section} % Number tables within sections (i.e. 1.1, 1.2, 2.1, 2.2 instead of 1, 2, 3, 4)

\setlength\parindent{0pt} % Removes all indentation from paragraphs

\setlist{noitemsep} % No spacing between list items

%%% Операторы всякие:
\DeclareMathOperator{\arsec}{arsec}
\DeclareMathOperator{\arcsch}{arcsch}
\DeclareMathOperator{\arcosh}{arcosh}
\DeclareMathOperator{\arsinh}{arsinh}
\DeclareMathOperator{\artanh}{artanh}
\DeclareMathOperator{\arsech}{arsech}
\DeclareMathOperator{\grad}{grad}
\DeclareMathOperator{\Log}{Log}
\DeclareMathOperator{\Arg}{Arg}
\renewcommand{\Im}{\mathop{\mathrm{Im}}\nolimits}
\renewcommand{\Re}{\mathop{\mathrm{Re}}\nolimits}
\DeclareMathOperator{\arcoth}{arcoth}
\usepackage{setspace}
%%% Колонтитулы
\usepackage{fancyhdr}
\pagestyle{fancy}
\renewcommand{\sectionmark}[1]{\markright{\thesection\ #1}}

\fancyhead[LE,RO]{\thepage}
\fancyhead[LO]{\rightmark}
\fancyhead[RE]{\leftmark}


%----------------------------------------------------------------------------------------
%	SECTION TITLES
%----------------------------------------------------------------------------------------

%\sectionfont{\vspace{6pt}\centering\normalfont\scshape} % \section{} styling
%\subsectionfont{\normalfont\bfseries} % \subsection{} styling
%\subsubsectionfont{\normalfont\itshape} % \subsubsection{} styling
%\paragraphfont{\normalfont\scshape} % \paragraph{} styling

\newcommand{\RNumb}[1]{\uppercase\expandafter{\romannumeral #1\relax}}

\renewcommand\thesection{\arabic{section}.}
\renewcommand\thesubsection{\thesection\arabic{subsection}}
\renewcommand\thesubsubsection{\thesubsection.\arabic{subsubsection}}
\renewcommand{\bf}{\textbf}
%----------------------------------------------------------------------------------------
%	HEADERS AND FOOTERS
%----------------------------------------------------------------------------------------
\DeclareMathOperator{\ord}{ord}
\DeclareMathOperator{\ld}{ld}
\DeclareMathOperator{\exi}{exi}
\DeclareMathOperator{\num}{num}
\DeclareMathOperator{\den}{den}
\DeclareMathOperator{\diam}{diam}
\DeclareMathOperator{\sign}{sign}
\DeclareMathOperator{\len}{len}
\DeclareMathOperator{\vp}{v.p.}
\DeclareMathOperator{\osc}{osc}
\newcommand{\divisible}{\mathop{\raisebox{-2pt}{\vdots}}}
\DeclareRobustCommand{\divby}{%
     \mathrel{\text{\vbox{\baselineskip.65ex\lineskiplimit0pt\hbox{.}\hbox{.}\hbox{.}}}}%
}
\newcommand{\eqdef}{\stackrel{\mathrm{def}}{=}}
\DeclareRobustCommand{\notdivby}{%
     \!\!\not\;\divby%
}
\DeclareMathOperator{\Int}{Int}
\DeclareMathOperator{\Cl}{Cl}
\DeclareMathOperator{\Fr}{Fr}
\newcommand{\Mod}[1]{\ (\mathrm{mod}\ #1)}



\addto{\captionsrussian}{\renewcommand{\abstractname}{АННОТАЦИЯ}}

\newcommand{\incfig}[1]{%
    \def\svgscale{1.5}
    \import{./figures/}{#1.pdf_tex}
}
\graphicspath{{pictures/}}
\DeclareGraphicsExtensions{.pdf,.png,.jpg, .jpeg, .tex}

\definecolor{myblue}{RGB}{72, 184, 178}
\definecolor{myblue1}{RGB}{0, 109, 167}
\usepackage{color}
\usepackage[colorlinks,urlcolor = blue, filecolor=blue,citecolor=blue, linkcolor = blue]{hyperref}
\setcounter{section}{1}
% Document
\begin{document}
\begin{center}
\section*{\textsc{Базовая теория чисел}}
\end{center}
    \subsection*{Делимость и её базовые свойства.}
    \begin{enumerate}[start=1,label={\bfseries \arabic*.}]
        \item Найдите все такие $a \in \mathbb{N}$, что $(a^2 + a - 1) \divisible (a - 2)$.

        \item Докажите, что произведение любых пяти последовательных натуральных чисел делится на 30.

        \item Придумайте 5 различных натуральных чисел, сумма которых делится на каждое из них.

        \item Докажите, что дробь $\cfrac{6n + 7}{10n + 12}$ несократима не при каких натуральных $n$.

        \item Произведение двух чисел, каждое из которых делится на 10, равно 1000. Найдите сумму этих чисел.

    \end{enumerate}
    \subsection*{Деление с остатком}
    \begin{enumerate}[start=6,label={\bfseries \arabic*.}]
        \item Найдите остаток числа $2011 \cdot 2012 + 2013^2$ при делении на 7.

        \item Докажите, что квадраты натуральных чисел не дают остатка 2 от деления на 3.

        \item Найдите последнюю цифру числа $2^2012$.

        \item Найдите четырёхзначное число, которое при делении на 131 даёт остаток 112, а при делении на 132 даёт остаток 98.

        \item  Натуральные числа $x, y, z$ образуют пифагорову тройку, то есть $x^2 + y^2 = z^2$. Докажите, что $x y z \divisible 60$.
    \end{enumerate}
    \subsection*{Сравнения по модулю}
    \begin{enumerate}[start=11,label={\bfseries \arabic*.}]
        \item Докажите, что число $96^{19} + 32^{13} - 8 \cdot 73^{16}$ делится на 10.
        \item Найдите остаток от деления $26^{36}$ на 7.

        \item Докажите, что $16^{2014} + 33^{2015}$ делится на 17.

        \item При каких натуральных $n$ число $2^n - 1$ делится на 7?

        \item Докажите, что $\forall n \in \mathbb{N} \ 37^{n + 2} + 16^{n + 1} + 23^{n} \divisible 7$.
    \end{enumerate}
    \subsection*{Десятичная запись числа. Признаки делимости.}
    \begin{enumerate}[start=16,label={\bfseries \arabic*.}]
        \item Цифры двузначного числа поменяли местами, после чего вычли полученное двузначное число из исходного. Докажите, что полученная разность делится на 9.

        \item Двузначное число умножили на удвоенное произведение его цифр. Получилось 2016. Найдите исходное число.

        \item Пусть $a, b, c, d$~--- различные цифры. Докажите, что $\overline{cdcdcdcd}$ не делится на $\overline{aabb}$.

        \item Существует ли натуральное число, которое при зачёркивании первой слева цифры уменьшается ровно в 2011 раз?

        \item Найдите все двузначные числа, которые равны сумме своей цифры десятков и квадрата цифры, стоящей в разряде единиц.

        \item Сколько существует двузначных чисел, которые ровно в 9 раз больше суммы своих цифр? Сколько существует таких трёхзначных чисел?

        \item Используя все цифры от 1 до 9 по одному разу, составьте наибольшее девятизначное число, делящееся на 11.

        \item Чтобы открыть сейф, нужно ввести код~--- семизначное число из двоек и троек. Известно, что в кодовом числе двоек больше, чем троек. Кроме того, известно, что кодовое число делится на 3 и на 4. Найдите код сейфа.

        \item Докажите, что число, состоящее из 100 нулей, 100 единиц и 100 двоек, не может быть квадратом натурального числа.
    \end{enumerate}
    \subsection*{НОК. НОД. Алгоритм Евклида. }
    \begin{enumerate}[start=25,label={\bfseries \arabic*.}]
        \item Найдите при помощи алгоритма Евклида $\gcd(2576, 154)$.

        \item Найдите все пары натуральных чисел $(a, b)$, для которых $\gcd(a, b) = 13, \ \lcm(a, b) = 78$.

        \item Найдите $\gcd(27, 96)$, а также, линейное представление $\gcd(27, 96)$.

        \item Найдите наибольший общий делитель всех чисел вида $p^2 - 1$, где $p$~--- простое число, большее 3 и меньшее 2010.

        \item Известно, что дробь $\frac{a}{b}$, где $a, b \in \mathbb{N}$~--- несократима. Докажите, что дробь $\frac{2a + b}{5a + 3b}$ несократима.
    \end{enumerate}
    \subsection*{Диофантовы уравнения. }
    \begin{enumerate}[start=30,label={\bfseries \arabic*.}]
        \item Решите в целых числах уравнение $3x + 2y = 7$.

        \item У осьминога восемь ног, а у морской звезды 5. Сколько в аквариуме тех и других, если у всех них всего 39 ног.

        \item Имеются контейнеры двух видов: по 130 килограмм и 160 килограмм. Сколько было контейнеров первого вида и сколько второго вида,
              если все вместе они весят 3 тонны. Укажите все решения.

        \item Остаток от деления некоторого натурального числа $n$ на 6 равен $n$, а остаток от деления $n$ на 15 равен 7. Чему равен
              остаток от деления $n$ на 30?
    \end{enumerate}
    \subsection*{Основная теорема арифметики}
    \begin{enumerate}[start=34,label={\bfseries \arabic*.}]
        \item Разложите на простые множители числа 111, 1111, 11111, 111111, 111111, 1111111.

        \item  В конце четверти Вовочка выписал подряд в строчку свои текущие отметки по пению и поставил между некоторыми из них знак умножения. Произведение получившихся чисел оказалось равным 2007. Какая отметка выходит у Вовочки в четверти по пению? («Колов» учительница пения не ставит.)

        \item Укажите пять целых положительных чисел, сумма которых равна 20, а произведение~--- 420.

        \item Докажите, что число является квадратом натурального числа тогда и только тогда, когда у него нечетное число делителей.

        \item Найдите все натуральные числа $n$ от 1 до 100 такие, что если перемножить все делители
              числа $n$ (включая $1$ и $n$), мы получим число $n^3$.

        \item Множество $A$ состоит из различных натуральных чисел. Количество чисел в $A$ больше семи. Наименьшее общее кратное всех чисел из  $A$ равно 210. Для любых двух чисел из $A$ их наибольший общий делитель больше единицы. Произведение всех чисел из $A$ делится на 1920 и не является квадратом никакого целого числа.
              Найти числа, из которых состоит $A$.
    \end{enumerate}
    \subsection*{Следствия из основной теоремы арифметики.}
    \begin{enumerate}[start=40,label={\bfseries \arabic*.}]
        \item Сколько двоек присутствует в разложении на простые множители числа $20!$?

        \item Найдите количество натуральных делителей числа $56^n$.

        \item Найдите наименьшее натуральное число $n$, для которого 1999! делится на $34^n$.

        \item Найдите все натуральные числа, которые делятся на 42 и и имеют ровно 42 натуральных делителей.

        \item Найти количество и сумму всех натуральных чисел, не превосходящих 1000 и имеющих нечетное число делителей.
    \end{enumerate}
    \subsection*{НОК. НОД. (2). Малая теорема Ферма.}
    \begin{enumerate}[start=45,label={\bfseries \arabic*.}]
        \item Покажите, что $13^{176} - 1$ делится на $89$.

        \item Найдите все пары натуральных чисел, разность которых равна 66, а их НОК равен 360.

        \item Найдите $\gcd(2^{30} - 1, 2^{40} - 1)$.

        \item Найтуральные числа $m$ и $n$ взаимнопросты. Какие значения может принимать НОД чисел $4m + 3n$ и $6m + 5n$,

        \item  Вычислите  $2010^{2011} \pmod{57}$.
    \end{enumerate}
\end{document}