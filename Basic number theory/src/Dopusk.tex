%! Author = user
%! Date = 12.07.2022

%! Author = matheusz
%! Date = 14.07.2021

% Preamble
\documentclass[12pt]{article}

% Packages
\usepackage[left=1.1cm,right=1.1cm,
    top=2cm,bottom=2cm,bindingoffset=0cm]{geometry}
\usepackage{tipa}
\usepackage{euscript}
\usepackage{mathrsfs}
%\usepackage{txfonts}
\usepackage{textcomp}
\usepackage{cmap}
\usepackage{mathtext}
\usepackage[T2A]{fontenc}
\usepackage[english,russian]{babel}
\usepackage{amsfonts}
\usepackage{amsmath,amsfonts,amssymb,amsthm,mathtools} % AMS
\usepackage{icomma} % "Умная" запятая: $0,2$ --- число, $0, 2$ --- перечисление
\usepackage{fancybox,fancyhdr} %this packages provides fancy up and bottom of page
\pagestyle{fancy}
\usepackage{setspace}
%\полуторный интервал
\fancyhead{}
\fancyhead[LE,RO]{\thepage}
\fancyhead[CO]{Теория делимости, допуск}
\fancyhead[LO]{ЛНШ ЛНМО}
\newcommand{\divisible}{\mathop{\raisebox{-2pt}{\vdots}}}
\fancyhead[CE]{}
\usepackage{enumitem}
\fancyfoot{}
\usepackage{import}
\usepackage{xifthen}
\usepackage{pdfpages}
\usepackage{transparent}
\usepackage{caption}
\usepackage[condensed,math]{anttor}
\usepackage[T1]{fontenc}

\begin{document}
    \textbf{\textsc{Ведунов Пётр}}
    \begin{itemize}
        \item Остатки.
        \item Сравнения по модулю.
        \item Десятичная запись.
        \item НОД и НОК.
        \item Диофантовы уравнения.
        \item КТО.
    \end{itemize}
    \textbf{\textsc{Долгопольский Тимофей}}
    \begin{itemize}
        \item Остатки.
        \item Сравнения по модулю.
        \item Десятичная запись числа.
    \end{itemize}
    \textbf{\textsc{Казин Савелий}}
    \begin{itemize}
        \item Остатки.
        \item Сравнения по модулю.
        \item Десятичная запись числа.
        \item Диофантовы уравнения.
        \item КТО.
        \item Функция Эйлера.
    \end{itemize}
    \textbf{\textsc{Кривочеев Платон}}
    \begin{itemize}
        \item Остатки.
        \item Сравнения по модулю.
        \item Десятичная запись числа.
        \item Диофантовы уравнения.
        \item КТО.
        \item Функция Эйлера.
    \end{itemize}
    \textbf{\textsc{Тараканов Георгий}}
    \begin{itemize}
        \item Остатки.
        \item Сравнения по модулю.
        \item Десятичная запись числа.
        \item НОД и НОК.
        \item Диофантовы уравнения.
        \item ОТА.
        \item КТО.
    \end{itemize}
    \textbf{\textsc{Титов Олег}}
    \begin{itemize}
        \item Остатки.
        \item Сравнения по модулю.
        \item НОД и НОК.
        \item Диофантовы уравнения.
        \item ОТА.
        \item Функция Эйлера.
    \end{itemize}
    \textbf{\textsc{Шутова Дарья}}
    \begin{itemize}
        \item Сравнения по модулю.
    \end{itemize}
    \textbf{\textsc{Харченко Михаил}}
    \begin{itemize}
        \item Сравнения по модулю.
        \item Десятичная запись.
        \item НОД и НОК.
        \item ОТА.
        \item Функция Эйлера.
    \end{itemize}
    \textbf{\textsc{Афанасьев Иван}}
    \begin{itemize}
        \item Сравнения по модулю.
        \item Десятичная запись числа.
        \item НОД и НОК.
        \item КТО.
    \end{itemize}
    \textbf{\textsc{Ильюшин Тимофей}}
    \begin{itemize}
        \item КТО.
        \item Функция Эйлера.
    \end{itemize}
    \textbf{\textsc{Рыбаков Иван}}
    \begin{itemize}
        \item КТО.
    \end{itemize}

    \newpage
    \begin{center}
        \section*{Теория делимости.}
        \subsection*{Допуск к экзамену.}
    \end{center}
    \subsection*{Остатки.}
        Найдите остаток от деления на 17 числа $2^{1999} + 1$.
    \subsection*{Сравнения по модулю.}
        Докажите, что $7^{7^{7^{7^{7}}}} - 7^{7^{7^{7}}} \divisible 10$.
    \subsection*{НОД и НОК.}
        На какие натуральные числа можно сократить дробь
        \[ \frac{3m - n}{5n + 2m}, \text{ если } \gcd(m, n)  = 1?\]
    \subsection*{Десятичная запись числа.}
        Докажите, что $\overline{abcd} \divisible 99$ тогда и только тогда, когда $\overline{ab} + \overline{cd} \divisible 99$.
    \subsection*{Диофантовы уравнения}
        Решите в целых числах уравнение
        \[ 34x - 21y = 1.\]
        (Укажите все решения.)
    \subsection*{Основная теорема арифметики.}
        Натуральное число $n$ имеет два простых делителя, общее количество делителей числа равно 6, а их суииа 28. Найдите само число.
    \subsection*{Китайская теорема об остатках.}
    Укажите все целые числа $x$, удовлетворяющие системе
    \[ \begin{cases} x \equiv 3 \pmod{5} \\ x \equiv 7 \pmod{17} \end{cases}.\]
    \subsection*{Функция Эйлера.}
    \pagestyle{empty}
\end{document}