%! Author = matheusz
%! Date = 14.07.2021

% Preamble
\documentclass[12pt]{article}

% Packages
\usepackage[left=1.1cm,right=1.1cm,
    top=2cm,bottom=2cm,bindingoffset=0cm]{geometry}
\usepackage{tipa}
\usepackage{euscript}
\usepackage{mathrsfs}
%\usepackage{txfonts}
\usepackage{textcomp}
\usepackage{cmap}
\usepackage{mathtext}
\usepackage[T2A]{fontenc}
\usepackage[english,russian]{babel}
\usepackage{amsfonts}
\usepackage{amsmath,amsfonts,amssymb,amsthm,mathtools} % AMS
\usepackage{icomma} % "Умная" запятая: $0,2$ --- число, $0, 2$ --- перечисление
\usepackage{fancybox,fancyhdr} %this packages provides fancy up and bottom of page
\pagestyle{fancy}
\usepackage{setspace}
%\полуторный интервал
\fancyhead{}
\fancyhead[LE,RO]{\thepage}
\fancyhead[CO]{План курса}
\fancyhead[LO]{Летний лагерь ЛНМО}
\fancyhead[CE]{}
\usepackage{enumitem}
\fancyfoot{}
\usepackage{import}
\usepackage{xifthen}
\usepackage{pdfpages}
\usepackage{transparent}
\usepackage{caption}
\usepackage[condensed,math]{anttor}
\usepackage[T1]{fontenc}


\begin{document}
\begin{center}
    \section*{Алгебраическая геометрия и теория чисел.}
    \subsection*{\textsc{М. Магин}}
\end{center}

\begin{enumerate}
    \item Нормированные поля и нормы на $\mathbb{Q}$.
    \item Кольцо $p$-адических чисел.
    \item Локализация, поле частных.
    \item Поле $p$-адических чисел.
    \item Пополнение метрического пространства по норме. Поле $p$-адических.
    \item Представители Тейхмюллера в $\mathbb{Z}_p$ и $\mathbb{Q}_p$.
    \item Приложения $p$-адических чисел к решению сравнений. Лемма Гензеля.
    \item Квадратичные формы и квадрики. Рациональные параметризации квадрик.
    \item Теорема Минковского-Хассе, доказательство для случая $n = 3$.
    \item Проективные пространства, проективная плоскость. Однородные координаты.
    \item Проективные квадрики.
    \item Бесконечноудалённые точки и конические сечения на $\mathbb{R}P^2$.
    \item Кубические кривые и групповая структура на них.
    \item Группа рациональных точек на эллиптической кривой.
    \item Экскурс об эллпитических кривых. Целые точки на эллиптических кривых.
\end{enumerate}

\end{document}
