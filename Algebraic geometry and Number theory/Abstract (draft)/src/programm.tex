%! Author = matheusz
%! Date = 14.07.2021

% Preamble
\documentclass[12pt]{article}

% Packages
\usepackage[left=1.1cm,right=1.1cm,
    top=2cm,bottom=2cm,bindingoffset=0cm]{geometry}
\usepackage{tipa}
\usepackage{euscript}
\usepackage{mathrsfs}
%\usepackage{txfonts}
\usepackage{textcomp}
\usepackage{cmap}
\usepackage{mathtext}
\usepackage[T2A]{fontenc}
\usepackage[english,russian]{babel}
\usepackage{amsfonts}
\usepackage{amsmath,amsfonts,amssymb,amsthm,mathtools} % AMS
\usepackage{icomma} % "Умная" запятая: $0,2$ --- число, $0, 2$ --- перечисление
\usepackage{fancybox,fancyhdr} %this packages provides fancy up and bottom of page
\pagestyle{fancy}
\usepackage{setspace}
%\полуторный интервал
\fancyhead{}
\fancyhead[LE,RO]{\thepage}
\fancyhead[CO]{План курса}
\fancyhead[LO]{Летний лагерь ЛНМО}
\fancyhead[CE]{}
\usepackage{enumitem}
\fancyfoot{}
\usepackage{import}
\usepackage{xifthen}
\usepackage{pdfpages}
\usepackage{transparent}
\usepackage{caption}
\usepackage[condensed,math]{anttor}
\usepackage[T1]{fontenc}


\begin{document}
\begin{center}
    \section*{Алгебраическая геометрия и теория чисел.}
    \subsection*{\textsc{М. Магин}}
\end{center}

\begin{enumerate}
    \item Нормированные поля и нормы на $\mathbb{Q}$. Связь неравенства треугольника и ультраметрического неравенства.
    \item Кольцо целых $p$-адических чисел. Мотивирующий пример про $\sqrt{2}$, определение $\mathbb{Z}_p$, каноническая форма целого $p$-адического числа.
    \item Мультипликативная группа $\mathbb{Z}_p$, представление элемента в виде произведения степени простого на обратимый.
    \item Локализация кольца и поле частных. Примеры локализаций.
    \item Поле $p$-адических чисел, как поле частных $\mathbb{Z}_p$.
    \item Сравнения в поле $\mathbb{Q}_p$.
    \item Сходимость в поле $p$-адических чисел. Целое $p$-адическое число является пределом задающей его последовательности. Критерий Коши. Критерий сходимости через разность соседних членов.
    \item Ряды в поле $\mathbb{Q}_p$. Критерий сходимости ряда с $p$-адическими членами. Представление $p$-адического числа ввиде ряда.
    \item Метод касательных Ньютона. Приложения $p$-адических чисел к решению сравнений. Лемма Гензеля.
    \item Квадратичные формы и квадрики. Рациональные параметризации квадрик, пример с пифагоровыми тройками.
    \item Теорема Минковского-Хассе, доказательство для случая $n = 3$.
    \item Проективные пространства, проективная плоскость (постоение через разные модели). Однородные координаты, проективные преобразования.
    \item Проективные квадрики.
    \item Бесконечноудалённые точки и конические сечения на $\mathbb{R}P^2$.
    \item Кубические кривые и групповая структура на них.
    \item Группа рациональных точек на эллиптической кривой.
    \item Экскурс об эллпитических кривых. Целые точки на эллиптических кривых.
\end{enumerate}

\end{document}
