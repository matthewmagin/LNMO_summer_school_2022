%! Author = matheusz
%! Date = 14.07.2021

% Preamble
\documentclass[12pt]{article}

% Packages
\usepackage[left=1.1cm,right=1.1cm,
    top=2cm,bottom=2cm,bindingoffset=0cm]{geometry}
\usepackage{tipa}
\usepackage{euscript}
\usepackage{mathrsfs}
%\usepackage{txfonts}
\usepackage{textcomp}
\usepackage{cmap}
\usepackage{mathtext}
\usepackage[T2A]{fontenc}
\usepackage[english,russian]{babel}
\usepackage{amsfonts}
\usepackage{amsmath,amsfonts,amssymb,amsthm,mathtools} % AMS
\usepackage{icomma} % "Умная" запятая: $0,2$ --- число, $0, 2$ --- перечисление
\usepackage{fancybox,fancyhdr} %this packages provides fancy up and bottom of page
\pagestyle{fancy}
\usepackage{setspace}
%\полуторный интервал
\fancyhead{}
\fancyhead[LE,RO]{\thepage}
\fancyhead[CO]{Вопросы к экзамену}
\fancyhead[LO]{ЛНШ ЛНМО}
\fancyhead[CE]{}
\usepackage{enumitem}
\fancyfoot{}
\usepackage{import}
\usepackage{xifthen}
\usepackage{pdfpages}
\usepackage{transparent}
\usepackage{caption}
\usepackage[condensed,math]{anttor}
\usepackage[T1]{fontenc}


\begin{document}
\begin{center}
    \section*{Алгебраическая геометрия и теория чисел.}
    \subsection*{Вопросы к письменному экзамену. }
    \subsubsection*{\textsc{М. И. Магин}}
\end{center}

\begin{enumerate}
    \item Нормированные поля: основные определения. Теорема о связи неравенства треугольника и ультраметрического неравенства.

    \item Построение кольца $p$-адических чисел: определение, арифметические операции в кольце, канонический способ задания целого $p$-адического числа.

    \item Мультипликативная группа кольца $\mathbb{Z}_p$.

    \item Теорема о представлении целого $p$-адического числа в виде $p^{\ell} \cdot \varepsilon$.

    \item Локализация кольца в мультипликативном подмножестве: определение, универсальное свойство, начало построения (утверждение про отношение эквивалентности).

    \item Локализация кольца в мультипликативном подмножестве: конец построения~--- действия в локализации, корректность свойств арифметических операций, универсальное свойство.

    \item Поле $\mathbb{Q}_p$ как поле частных $\mathbb{Z}_p$. Представление $p$-адического числа в виде $p^{\ell} \cdot \varepsilon$.

    \item Сходимость в поле $p$-адических чисел, теоремы о действиях с пределами (доказательство на примере одного свойства на выбор).

    \item Сравнения по модулю в кольце целых  $p$-адических чисел. Теорема о сравнениях в кольце $\mathbb{Z}_p$.

    \item $p$-адическое число как предел определяющей его последовательности.

    \item Лемма Больцано-Вейерштрасса (без доказательства), критерий Коши в поле $p$-адических чисел.

    \item Теорема о том, что сходимость $\{ \xi_n \}$ равносильна тому, что $\lim\limits_{n \to \infty} \upsilon_p{(\xi_{n + 1} - \xi_n)} = \infty$.

    \item Ряды с $p$-адическими членами. Критерий сходимости рядов с $p$-адическими членами.

    \item $p$-адическое число, как сумма ряда. Аналогия между представлением вещественных чисел в десятичной записи и $p$-адическими числами.

    \item Метод касательных Ньютона.

    \item Лемма Гензеля.

    \item  Следствия из леммы Гензеля и переформулировки леммы Гензеля.

    \item Проективная плоскость и различные модели. Склейка многогранника, факторпространство $D^2$ или $S^2$, прямые на проективной плоскости.

    \item Проективная плоскость как евклидова плоскость с бесконечно удаленными точками. Прямые в такой модели. Бесконечно удалённая прямая.

    \item Определение проективных пространств, свойства, однородные координаты.

    \item Проективное пополнение $\mathbb{R}^n$.

    \item Проективные преобразования. Группа преобразований Мёбиуса проективной прямой $\mathbb{R}P^1$.

    \item Квадратичные функции и квадрики. Рациональные параметризации. Вывод формул для Пифагоровых троек.

    \item Билинейные формы и их матрицы. Квадратичные формы и соответствующие им  билинейные формы. Теорема о диагонализации (без доказательства).

    \item Сигнатура квадратичной формы. Закон инерции квадратичных форм (без доказательства). Классификация квадрик на прямой, проективизация.

    \item  Квадрики на плоскости: эллипс, парабола, гипербола (определение, принадлежность к квадрикам (можно либо для эллипса, либо для гиперболы)), проективизация.

    \item Проективная классификация квадрик.

    \item Аффинная классификация квадрик на плоскости.

    \item Принцип Минковского-Хассе для квадратичных форм (формулировка и основные моменты доказательства).

    \item Алгебраические кривые. Эллиптические кривые. Группа точек на эллиптической кривой (корректность, всего кроме ассоциативности).

    \item Ассоциативность сложения точек на эллиптической кривой.

\end{enumerate}

\end{document}
