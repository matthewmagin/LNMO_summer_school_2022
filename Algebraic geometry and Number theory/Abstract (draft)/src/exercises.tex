%! Author = Matthew
%! Date = 22.06.2022

% Preamble
\documentclass[11pt]{article}

% Packages
%----------------------------------------------------------------------------------------
%	PACKAGES AND OTHER DOCUMENT CONFIGURATIONS
%----------------------------------------------------------------------------------------

\usepackage{listings}
\usepackage{graphicx}
\usepackage{booktabs}
\usepackage{enumitem}
\usepackage[left=2cm,right=2cm,
    top=2cm,bottom=2cm,bindingoffset=0cm]{geometry}
\usepackage[utf8]{inputenc}
\usepackage[english,russian]{babel}
\usepackage{titling}
\usepackage{textcomp}
\usepackage{mathtext}
\usepackage{amsmath,amsfonts,amssymb,amsthm,mathtools}
\usepackage{icomma}
\usepackage{import}
\usepackage{amssymb, amsmath}
\usepackage{indentfirst}
\usepackage{moresize}
\usepackage{multicol}
\usepackage{dsfont}
\usepackage{xifthen}
\usepackage{pdfpages}
\usepackage{transparent}
\usepackage{caption}
\usepackage{epigraph}
\usepackage{xcolor}

\newtheorem{statement}{Statement}
\newtheorem{corollary}{Corollary}
\newtheorem{theorem}{Theorem}
\newtheorem{definition}{Definition}
\newtheorem{lemma}{Lemma}
\newtheorem{example}{Example}
\theoremstyle{remark}
\newtheorem{remark}{Remark}
\newtheorem{prop}{Property}




\numberwithin{equation}{section} % Number equations within sections (i.e. 1.1, 1.2, 2.1, 2.2 instead of 1, 2, 3, 4)
\numberwithin{figure}{section} % Number figures within sections (i.e. 1.1, 1.2, 2.1, 2.2 instead of 1, 2, 3, 4)
\numberwithin{table}{section} % Number tables within sections (i.e. 1.1, 1.2, 2.1, 2.2 instead of 1, 2, 3, 4)

\setlength\parindent{0pt} % Removes all indentation from paragraphs

\setlist{noitemsep} % No spacing between list items

%%% Операторы всякие:
\DeclareMathOperator{\arsec}{arsec}
\DeclareMathOperator{\arcsch}{arcsch}
\DeclareMathOperator{\arcosh}{arcosh}
\DeclareMathOperator{\arsinh}{arsinh}
\DeclareMathOperator{\artanh}{artanh}
\DeclareMathOperator{\arsech}{arsech}
\DeclareMathOperator{\grad}{grad}
\DeclareMathOperator{\Log}{Log}
\DeclareMathOperator{\Arg}{Arg}
\renewcommand{\Im}{\mathop{\mathrm{Im}}\nolimits}
\renewcommand{\Re}{\mathop{\mathrm{Re}}\nolimits}
\DeclareMathOperator{\arcoth}{arcoth}
\usepackage{setspace}
%%% Колонтитулы
\usepackage{fancyhdr}
\pagestyle{fancy}
\renewcommand{\sectionmark}[1]{\markright{\thesection\ #1}}

\fancyhead[LE,RO]{\thepage}
\fancyhead[LO]{\rightmark}
\fancyhead[RE]{\leftmark}


%----------------------------------------------------------------------------------------
%	SECTION TITLES
%----------------------------------------------------------------------------------------

%\sectionfont{\vspace{6pt}\centering\normalfont\scshape} % \section{} styling
%\subsectionfont{\normalfont\bfseries} % \subsection{} styling
%\subsubsectionfont{\normalfont\itshape} % \subsubsection{} styling
%\paragraphfont{\normalfont\scshape} % \paragraph{} styling

\newcommand{\RNumb}[1]{\uppercase\expandafter{\romannumeral #1\relax}}

\renewcommand\thesection{\arabic{section}.}
\renewcommand\thesubsection{\thesection\arabic{subsection}}
\renewcommand\thesubsubsection{\thesubsection.\arabic{subsubsection}}
\renewcommand{\bf}{\textbf}
%----------------------------------------------------------------------------------------
%	HEADERS AND FOOTERS
%----------------------------------------------------------------------------------------
\DeclareMathOperator{\ord}{ord}
\DeclareMathOperator{\ld}{ld}
\DeclareMathOperator{\exi}{exi}
\DeclareMathOperator{\num}{num}
\DeclareMathOperator{\den}{den}
\DeclareMathOperator{\diam}{diam}
\DeclareMathOperator{\sign}{sign}
\DeclareMathOperator{\len}{len}
\DeclareMathOperator{\vp}{v.p.}
\DeclareMathOperator{\osc}{osc}
\newcommand{\divisible}{\mathop{\raisebox{-2pt}{\vdots}}}
\DeclareRobustCommand{\divby}{%
     \mathrel{\text{\vbox{\baselineskip.65ex\lineskiplimit0pt\hbox{.}\hbox{.}\hbox{.}}}}%
}
\newcommand{\eqdef}{\stackrel{\mathrm{def}}{=}}
\DeclareRobustCommand{\notdivby}{%
     \!\!\not\;\divby%
}
\DeclareMathOperator{\Int}{Int}
\DeclareMathOperator{\Cl}{Cl}
\DeclareMathOperator{\Fr}{Fr}
\newcommand{\Mod}[1]{\ (\mathrm{mod}\ #1)}



\addto{\captionsrussian}{\renewcommand{\abstractname}{АННОТАЦИЯ}}

\newcommand{\incfig}[1]{%
    \def\svgscale{1.5}
    \import{./figures/}{#1.pdf_tex}
}
\graphicspath{{pictures/}}
\DeclareGraphicsExtensions{.pdf,.png,.jpg, .jpeg, .tex}

\definecolor{myblue}{RGB}{72, 184, 178}
\definecolor{myblue1}{RGB}{0, 109, 167}
\usepackage{color}
\usepackage[colorlinks,urlcolor = blue, filecolor=blue,citecolor=blue, linkcolor = blue]{hyperref}

% Document
\begin{document}
\begin{center}
        \section*{Алгебраическая геометрия и теория чисел}
    \end{center}
    \tableofcontents
    \newpage
    \subsection*{Правила сдачи.}

    Говорят, что решение задач серьезно помогает в закреплении курсов. \\

    После каждой лекции в этой в этом файле  репозитории \url{https://github.com/matthewmagin/LNMO_summer_school_2022}
    будет добавляться несколько задач с некоторой стоимостью (она указана справа от номера).\\

    Задачи не теряют цену \textbf{2 дня}, после баллы за задачу считаются, как $N \cdot 0,9^{t}$, где
    $t$~--- количество дней, на которые задача просрочена, а $N$~--- ценность задачи. \\

    \textbf{\textsc{Зачем вообще это решать:}}\\
    Наличие баллов за задачи даёт бонусы на экзамене. Тем, у кого баллов будет много
    (это будет несколько людей) можно будет не решать задачу на 5, остальным баллы будут учитываться в прицнипе при выставлении оценки.\\

    \textbf{\textsc{Приз:}}\\
    Тот, кто нарешает задач больше всех, получит приз от лектора.\\

    \textbf{\textsc{Как сдавать задачи:}}\\
    Сдавать задачи надо устно лектору (решения лучше пишите заранее).

    \begin{center}
        \textsc{Удачи!}
    \end{center}
    \newpage

    \subsection{ТЧ: напоминание, чтобы быть в форме}
    \begin{enumerate}[start=1,label={\bfseries \arabic*.}]
        \item \textcolor{blue}{(3б.)} Найдите $17^{26^{39}} \mod{330}$

        \item \textcolor{blue}{(5б.)} Найдите 4 последние цирфы числа $18^{1818}$.

    \end{enumerate}
    \subsection{Нормированное поле. Неархимедовы нормы.}
    \begin{enumerate}[start=1,label={\bfseries \arabic*.}]
        \item \textcolor{blue}{(3б.)} Докажите, что на конечном поле $\mathbb{F}_{p}$ не существует нетривиальной нормы.

        \item \textcolor{blue}{(2б.)} Докажите арифметические свойства пределов для $(F, \| \cdot \|)$.

        \item \textcolor{blue}{(2б.)} Докажите, что в неархимедовом нормированном поле любой шар с положительным радиусом является одновременно и открытым
              и замкнутым множеством.
    \end{enumerate}
    \subsection{p-адические числа.}
    \begin{enumerate}[start=1,label={\bfseries \arabic*.}]
        \item
    \end{enumerate}
    \subsection{p-адический анализ.}
    \begin{enumerate}[start=1,label={\bfseries \arabic*.}]
        \item
    \end{enumerate}
    \subsection{Проективная геометрия. Квадрики и проективные квадрики.}
    \begin{enumerate}
        \item Что будет, если вырезать (открытую окрестность) из $\mathbb{R}P^2$?

        \item Докажите, что любое проективное отображение $\mathbb{R}P^1 \to \mathbb{R}P^1$ является композицией не более двух инволюций.

        \item Докажите директориальное свойство эллипса (не совпадающего с  окружностью) и гиперболы. Директрисами эллипса $\frac{x^2}{a^2} + \frac{y^2}{b^2} = 1$ или гиперболы  $\frac{x^2}{a^2} - \frac{y^2}{b^2} = 1$ называются две прямые $x = \pm\frac{a}{\varepsilon}$, где число $\varepsilon$, называемое эксцентриситетом, 
определяется равенством $\varepsilon = \frac{c}{a}$ , где $2c$~—- расстояние между фокусами. Докажите, что отношение расстояний от точки эллипса (или гиперболы) до правого фокуса к расстоянию до правой директрисы равно эксцентриситету (то же верно и для левых фокуса и директрисы).

        \item Нарисуйте в окрестности бесконечно удаленной точки кубическую пораболу, заданную в декартовых координатах на Евклидовой плоскости
              уравнением $y = x^3$.

        \item Расклассифицируйте проективыне квадрики в $\mathbb{R}P^3$. \\
              \textsc{Замечание:} можно пользоваться классифицкаций квадрик в $\mathbb{R}^3$.
    \end{enumerate}
\end{document}