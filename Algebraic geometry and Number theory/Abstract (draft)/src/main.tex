%! Author = user
%! Date = 22.06.2022

% Preamble
\documentclass[11pt]{article}

% Packages
%----------------------------------------------------------------------------------------
%	PACKAGES AND OTHER DOCUMENT CONFIGURATIONS
%----------------------------------------------------------------------------------------

\usepackage{listings}
\usepackage{graphicx}
\usepackage{booktabs}
\usepackage{enumitem}
\usepackage[left=2cm,right=2cm,
    top=2cm,bottom=2cm,bindingoffset=0cm]{geometry}
\usepackage[utf8]{inputenc}
\usepackage[condensed,math]{anttor}
\usepackage[T1]{fontenc}
\usepackage[english,russian]{babel}
\usepackage{titling}
\usepackage{textcomp}
\usepackage{mathtext}
\usepackage{amsmath,amsfonts,amssymb,amsthm,mathtools}
\usepackage{icomma}
\usepackage{import}
\usepackage{amssymb, amsmath}
\usepackage{indentfirst}
\usepackage{moresize}
\usepackage{multicol}
\usepackage{dsfont}
\usepackage{xifthen}
\usepackage{pdfpages}
\usepackage{transparent}
\usepackage{caption}
\usepackage{epigraph}
\usepackage{xcolor}

\newtheorem{statement}{Statement}
\newtheorem{corollary}{Corollary}
\newtheorem{theorem}{Theorem}
\newtheorem{definition}{Definition}
\newtheorem{lemma}{Lemma}
\newtheorem{example}{Example}
\theoremstyle{remark}
\newtheorem{remark}{Remark}
\newtheorem{prop}{Property}




\numberwithin{equation}{section} % Number equations within sections (i.e. 1.1, 1.2, 2.1, 2.2 instead of 1, 2, 3, 4)
\numberwithin{figure}{section} % Number figures within sections (i.e. 1.1, 1.2, 2.1, 2.2 instead of 1, 2, 3, 4)
\numberwithin{table}{section} % Number tables within sections (i.e. 1.1, 1.2, 2.1, 2.2 instead of 1, 2, 3, 4)

\setlength\parindent{0pt} % Removes all indentation from paragraphs

\setlist{noitemsep} % No spacing between list items

%%% Операторы всякие:
\DeclareMathOperator{\arsec}{arsec}
\DeclareMathOperator{\arcsch}{arcsch}
\DeclareMathOperator{\arcosh}{arcosh}
\DeclareMathOperator{\arsinh}{arsinh}
\DeclareMathOperator{\artanh}{artanh}
\DeclareMathOperator{\arsech}{arsech}
\DeclareMathOperator{\grad}{grad}
\DeclareMathOperator{\Log}{Log}
\DeclareMathOperator{\Arg}{Arg}
\renewcommand{\Im}{\mathop{\mathrm{Im}}\nolimits}
\renewcommand{\Re}{\mathop{\mathrm{Re}}\nolimits}
\DeclareMathOperator{\arcoth}{arcoth}
\usepackage{setspace}
%%% Колонтитулы
\usepackage{fancyhdr}
\pagestyle{fancy}
\renewcommand{\sectionmark}[1]{\markright{\thesection\ #1}}

\fancyhead[LE,RO]{\thepage}
\fancyhead[LO]{\rightmark}
\fancyhead[RE]{\leftmark}


%----------------------------------------------------------------------------------------
%	SECTION TITLES
%----------------------------------------------------------------------------------------

%\sectionfont{\vspace{6pt}\centering\normalfont\scshape} % \section{} styling
%\subsectionfont{\normalfont\bfseries} % \subsection{} styling
%\subsubsectionfont{\normalfont\itshape} % \subsubsection{} styling
%\paragraphfont{\normalfont\scshape} % \paragraph{} styling

\newcommand{\RNumb}[1]{\uppercase\expandafter{\romannumeral #1\relax}}

\renewcommand\thesection{\arabic{section}.}
\renewcommand\thesubsection{\thesection\arabic{subsection}}
\renewcommand\thesubsubsection{\thesubsection.\arabic{subsubsection}}
\renewcommand{\bf}{\textbf}
%----------------------------------------------------------------------------------------
%	HEADERS AND FOOTERS
%----------------------------------------------------------------------------------------
\DeclareMathOperator{\ord}{ord}
\DeclareMathOperator{\ld}{ld}
\DeclareMathOperator{\exi}{exi}
\DeclareMathOperator{\num}{num}
\DeclareMathOperator{\den}{den}
\DeclareMathOperator{\diam}{diam}
\DeclareMathOperator{\sign}{sign}
\DeclareMathOperator{\len}{len}
\DeclareMathOperator{\vp}{v.p.}
\DeclareMathOperator{\osc}{osc}
\newcommand{\divisible}{\mathop{\raisebox{-2pt}{\vdots}}}
\DeclareRobustCommand{\divby}{%
     \mathrel{\text{\vbox{\baselineskip.65ex\lineskiplimit0pt\hbox{.}\hbox{.}\hbox{.}}}}%
}
\newcommand{\eqdef}{\stackrel{\mathrm{def}}{=}}
\DeclareRobustCommand{\notdivby}{%
     \!\!\not\;\divby%
}
\DeclareMathOperator{\Int}{Int}
\DeclareMathOperator{\Cl}{Cl}
\DeclareMathOperator{\Fr}{Fr}
\newcommand{\Mod}[1]{\ (\mathrm{mod}\ #1)}



\addto{\captionsrussian}{\renewcommand{\abstractname}{АННОТАЦИЯ}}

\newcommand{\incfig}[1]{%
    \def\svgscale{1.5}
    \import{./figures/}{#1.pdf_tex}
}
\graphicspath{{pictures/}}
\DeclareGraphicsExtensions{.pdf,.png,.jpg, .jpeg, .tex}

\definecolor{myblue}{RGB}{72, 184, 178}
\definecolor{myblue1}{RGB}{0, 109, 167}
\usepackage{color}
\usepackage[colorlinks,urlcolor = blue, filecolor=blue,citecolor=blue, linkcolor = blue]{hyperref}

% Document
\begin{document}
    \begin{center}
        \section*{Алгебраическая геометрия и теория чисел}
    \end{center}
    \tableofcontents
    \newpage

    \section{Нормированные поля}
    \subsection{Нормированное поле. Неархимедовы нормы.}
    Здесь и вдальнейшем будем полагать $F$ полем, хотя многие вещи работают и для кольца (а для области целостности существует
    единственное продолжение на поле частных).

    \begin{definition}\label{fieldnorm}
     Нормой (нормированием, абсолютным значением) на поле $F$  называют отображение $\| \cdot \|\colon F \to \mathbb{R}_{> 0}$,
        удовлетворяющее следующим свойствам:
        \begin{enumerate}
            \item $\| x \| = 0 \Leftrightarrow x = 0$.

            \item $\forall x, y \in F \ \| x y \| = \| x \| \| y \| $.

            \item $\exists C > 0\colon \forall x, y \in F\colon$
            \[ \| x + y \| \le \max(x, y) \]
        \end{enumerate}
        Пара $(F, \| \cdot \|)$ называется нормированным полем.
    \end{definition}
    \begin{remark}
        Тем, кто уже до этого видел определение нормы, это определение может показаться странным, так как обычно вместо третьего свойства
        требуют неравенство треугольника:
        \[ \forall x, y \in F \ \| x + y \| \le \| x \| + \| y \| \]
        Ясно, что третье свойство следует из неравенства треугольника с $C = 2$. Ниже мы покажем и обратную импликацию.
    \end{remark}

    Ясно, что любая норма задаёт метрику $d(x, y) = \| x - y \|$, а любая метрика индуцирует топологию стандартным образом.

    \begin{example}
        Если $F \le \mathbb{C}$, то подходит $| \cdot |$ (модуль комплексного числа). Если $F \le \mathbb{R}$ или $F \le \mathbb{Q}$, то подходит $| \cdot |$.
    \end{example}

    \begin{example}
        На любом поле можно ввести тривиальную норму (иногда соответствующую ей метрику называют метрикой лентяя):
        \[ \| x \| = \begin{cases} 0, x = 0 \\ 1, x \neq 0 \end{cases}\]
    \end{example}

    \begin{theorem}
        Если в определении \ref{fieldnorm} постоянная $C$ равна $2$, то норма удовлетворяет неравенству треугольника.
    \end{theorem}
    \begin{proof}

        Сначала отметим, что если $n, m \in \mathbb{N}, \ n \le 2^m$, то выполняется оценка:
        \[ \| x_1 + x_2 + \ldots  + x_n \| \le C^m \cdot \| \max\limits_{1 \le k \le n} \| x_k \| \]
        Тогда мы можем провести оценки следующим образом:
        \begin{multline*} \| x + y \|^n = \| (x + y)^n \| = \| \sum\limits_{k = 0}^{n} \binom{n}{k} x^k y^{n - k} \| \le 2(n + 1) \max\limits_{0 \le k \le n}\| \binom{n}{k} x^k y^{n - k} \| \le 2(n + 1) \max\limits_{0 \le k \le n}\left( 2 \binom{n}{k} \|x\|^k \| y \|^{n - k} \right) \le \\
        \le 4(n + 1) (\| x \| + \| y \| )^n \end{multline*}
        Преобразуем это неравенство
        \[ \left( \frac{\| x + y \| }{\| x \| + \| y \|} \right)^n \le 4(n + 1) \leftrightarrow  \frac{\| x + y \|}{\| x \| + \| y \| } \le 4^{\frac{1}{n}} \cdot (n + 1)^{\frac{1}{n}} \]
        В пределе при $n \to \infty$ получаем:
        \[ \frac{\| x + y \| }{\| x \| + \| y \| } \le 1 \Leftrightarrow \| x + y \| \le \| x \| + \| y \| \]
    \end{proof}

    \begin{remark}
        Пример $F = \mathbb{C}$ с нормой $\| \cdot \| = | \cdot |^{\alpha}, \ \alpha > 1$ показывает, что константу $C = 2$ нельзя улучшить.
    \end{remark}
    \begin{remark}
        Тем самым, мы показали, что норму можно понимать, как функтор из категории $\mathcal{F}ield$ в категорию $\mathcal{M}etr$.
    \end{remark}
    \begin{corollary}
        Норма непрерывна.
    \end{corollary}
    \begin{definition}
        Нормы, с постойнной $C = 1$ в определении \ref{fieldnorm} называют неархимедовыми. Нормы, не являющиеся неархимедовыми,
        называют архимедовыми.
    \end{definition}
    \begin{example}
        Тривиальная норма на любом поле $F$ является неархимедовой.
    \end{example}
    \begin{definition}
        Ясно, что любое $x \in \mathbb{Q}$ представимо в виде $x = p^n \cdot \frac{a}{b}$, где $a, b \in \mathbb{Z}, \ a \not\divisible p, \ a \not\divisible p$, а $n \in \mathbb{Z}$.
        В таком случае число $n$ называют $p$-адическим показателем числа $x$ и обозначают $\upsilon_p(x)$.
    \end{definition}
    \begin{definition} \textbf{(Самое важное)}\\
    Пусть $p$~--- простое число. Тогда норму
        \[ \| x \|_{p} = \begin{cases} 0, x = 0 \\ p^{-\upsilon(p)}, x \neq 0  \end{cases}\]
    на поле $\mathbb{Q}$ называют $p$-адической нормой.
    \end{definition}

    \begin{remark}
        Ясно, что подоходит $r^{-\upsilon_p(x)}$, где $r > 1$, но $p$ брать удобно, так как для $x \in \mathbb{Q}^{*}$ справедлива формула произведения
        \[ 1 = \prod\limits_{p} |x| \cdot  \| x \|_{p} \]
    \end{remark}
    \begin{lemma}
        Если норма неархимедова, то для $x, y\colon \| x \| \neq \| y \|$ выполняется $\| x + y \| = \max{\| x \|, \| y \| }$.
    \end{lemma}
    \begin{corollary}
        Рассмотрим $(F, \| \cdot \| )$, где норма $\| \cdot \|$ неархимедова. Тогда, если  $b \in B_r(a)$, то $B_r(a) = B_r(b)$.
     \end{corollary}
    \begin{corollary} (Забавное)\\
        Если на поле $F$ введена неархимедова норма $F$, то $\forall x, y, z \in F$ по крайней мере два числа из
        $\| x - y \|, \ \| x - z \|, \ \| y - z\| $ равны. \\
        Иными словами, в метрическом пространстве $(F, d)$ ($d(x, y) = \| x - y \| $) все треугольники равнобедренные.
    \end{corollary}

    \subsection{Эквивалентные нормы.}
    Пока не знаю, буду ли рассказывать.
    
    \subsection{Пополнение метрических пространств}

    \begin{definition}
    Пусть $(X, d_X)$ --- метрическое пространство, $\mathcal{F}(X)$ --- множество всех ограниченных функций из $X$ в $\mathbb{R}$. Тогда введём расстояние $d_{\infty}$ между функциями $f, g \in \mathcal{F}(X)$:
    \begin{equation*}
        d_{\infty}(f, g) \eqdef \sup\{|f(x) - g(x)|, x \in X\}
    \end{equation*}
    Заметим, что определение корректно, так как функции ограничены.
    \end{definition}

    \begin{lemma}

    $(\mathcal{F}(X), d_{\infty})$ --- метрическое пространство.

    \end{lemma}

    \begin{proof}

     Проверим три аксиомы метрики:
    \begin{enumerate}
        \item Пусть $f = g$. Тогда $|f(x) - g(x)| = 0$ для всякого $x \in X$, так что $d_{\infty}(f, g) = 0$. Если же наоборот $d_{\infty}(f, g) = 0$, то $0 \leq |f(x) - g(x)| \leq \sup = 0$, а значит $f(x) = g(x)$ для всех $x \in X$, что и означает $f \equiv g$.
        \item Так как $|f(x) - g(x)| = |g(x) - f(x)|$, то и $d_{\infty}(f, g) = d_{\infty}(g, f)$.
        \item Рассмотрим три ограниченные функции $f, g, h \in \mathcal{F}(X)$, и покажем, что
        \begin{equation*}
            d_{\infty}(f, g) + d_{\infty}(g, h) \geq d_{\infty}(f, h)
        \end{equation*}
        Мы знаем, что:
        \begin{equation*}
            \forall x \in X: |f(x) - g(x)| + |g(x) - h(x)| \geq |f(x) - h(x)|
        \end{equation*}
        в силу неравенства треугольника для стандартной метрики на $\mathbb{R}$. Для всякого $\varepsilon > 0$ мы можем взять $x_0$ такой, что $|f(x_0) - h(x_0)| \geq \sup\{|f(x) - h(x)|, x \in X\} - \varepsilon$. Получаем, что
        \begin{equation*}
            d_{\infty}(f, h) - \varepsilon = \sup\{|f(x) - h(x)|, x \in X\} - \varepsilon \leq |f(x_0) - h(x_0)| \leq
        \end{equation*}
        \begin{equation*}
            \leq |f(x_0) - g(x_0)| + |g(x_0) - h(x_0)| \leq d_{\infty}(f, g) + d_{\infty}(g, h)
        \end{equation*}
        а раз это верно для любого $\varepsilon > 0$, то искомое неравенство доказано.
    \end{enumerate}
    \end{proof}

    \begin{lemma}
    $\mathcal{F}(X)$~--- полно.
    \begin{proof}
    Пусть $f_{n}$~--- фундаментальная последовательность функций. Тогда $\forall x_0 \in X: \{f_{n}(x_0)\}$ --- также фундаментальная последовательность, так как $|f_{n}(x_0) - f_{m}(x_0)| \leq \sup\{|f_{n}(x) - f_{m}(x)|, x \in X\}$. Следовательно,
    \begin{equation*}
        \forall x_0 \in X: \exists \lim_{n \to \infty} f_{n}(x_0)
    \end{equation*}
    и сходимость по всем точкам равномерна, так как не зависит от выбора точки $x_0$. Иными словами,
    \begin{equation*}
        \exists f(x): \forall \varepsilon > 0: \exists N: \forall n > N: d_{\infty}(f_n, f) < \varepsilon
    \end{equation*}
    где $f(x_0)$ определяется как предел $ \lim_{n \to \infty} f_{n}(x_0)$. Так что $f(x)$ --- функция, являющаяся пределом искомой последовательности функций.
    \end{proof}
    \end{lemma}

    \begin{definition}
    Пусть $(X, d_X)$~--- метрическое пространство, $\mathcal{F}(X)$~--- множество ограниченных функций из $X$ в $\mathbb{R}$. Построим изометрическое вложение $k: X \to \mathcal{F}(X)$ следующим образом:
    \begin{enumerate}
        \item Если $X$~--- ограничено, то определим $k(x) = d_x$, где
        \begin{equation*}
            \forall y \in X: d_x(y) \eqdef d_X(x, y)
        \end{equation*}
        Функция $d_x$ ограничена, так как $X$ ограничено. Заметим также, что
        \begin{equation*}
            d_{\infty}(d_x, d_y) = \sup_{z} |d_x(z) - d_y(z)| = \sup_z \big( d_X(x, z) - d_X(z, y) \big) \leq d_X(x, y)
        \end{equation*}
        однако равенство достигается при $z = y$, так что $d_{\infty}(d_x, d_y) = d_X(x, y)$, а значит вложение изометрическое.
        \item Пусть $X$, возможно, не ограничено. Тогда определим $k(x) = d_x - d_{x_0}$ для некоторой фиксированной точки $x_0 \in X$, где
        \begin{equation*}
            \forall y \in X: \big(k(x)\big)(y) \eqdef d_x(y) - d_{x_0}(y) = d_X(x, y) - d_X(y, x_0)
        \end{equation*}
        что есть ограниченная функция, так как $\forall y \in X: d_X(x, y) - d_X(y, x_0) \leq d_X(x, x_0)$. \\
        Заметим, что это аналогичным образом будет изометрическим вложением:
        \begin{equation*}
            d_{\infty}(d_x - d_{x_0}, d_y - d_{x_0}) = \sup_{z} |d_x(z) - d_{x_0}(z) - d_y(z) + d_{x_0}(z)| =
        \end{equation*}
        \begin{equation*}
            = \sup_z \big( d_X(x, z) - d_X(z, y) \big) \leq d_X(x, y)
        \end{equation*}
        где равенство достигается при $z = y$.
    \end{enumerate}
    \end{definition}

    \begin{Theorem}
    Любое метрическое пространство $(X, d_X)$ имеет пополнение $(\overline{X}, d_{\overline{X}})$, то есть такое метрическое пространство $\overline{X}$, что выполнено:
    \begin{enumerate}
        \item $X \subseteq \overline{X}$
        \item $X$ --- всюдю плотно в $\overline{X}$
        \item $d_{\overline{X}}|_X = d_X$, то есть вложение из $X$ в $\overline{X}$ является изометрическим
        \vspace{3pt}
        \item $(\overline{X}, d_{\overline{X}})$ --- полно.
    \end{enumerate}
    \end{Theorem}
    \begin{proof}
    Возьмём изометрическое вложение Куратовского $k: X \to \mathcal{F}(X)$, и возьмём его замыкание в топологическом пространстве $\mathcal{F}(X)$ с топологией, индуцированной метрикой $d_{\infty}$ --- назовём это замыкание $\overline{X}$. Заметим, что
    \begin{enumerate}
        \item $X \subseteq \overline{X}$ естественным образом
        \item $X$ всюду плотно в $\overline{X}$, так как любое множество всюду плотно в своём замыкании
        \item Вложение $X$ в $\overline{X}$ изометрическое, так как оно изометрическое и во всё пространство $\mathcal{F}(x)$
        \item $\overline{X}$ полно как замкнутое подмножество полного пространства.
    \end{enumerate}
    \end{proof}


    \begin{remark}
    Пополнение метрического пространства \textbf{единственно} с точностью до изометрии.
    \end{remark}

    \begin{remark}
    Выражение $X \subseteq \overline{X}$ тоже подразумевается с точностью до изометрии.
    \end{remark}
    \subsection{Пополнение нормированного поля.}
    Теперь мы умеем пополнять метрические пространства, но нам никто не гарантирует, что при пополнении поля по норме получится поле.

\end{document}
